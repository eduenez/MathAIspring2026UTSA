\documentclass[12pt]{article}

% --- Packages ---
\usepackage[utf8]{inputenc}
\usepackage[T1]{fontenc}
\usepackage[margin=1in]{geometry}
\usepackage{amsmath, amsfonts, amssymb, amsthm}
\usepackage{graphicx}
\usepackage{hyperref}
\usepackage{enumitem}
\usepackage{listings}
\usepackage{xcolor}
\usepackage{tcolorbox} % For the modern "Problem" boxes
\usepackage{fancyhdr}  % For headers and footers

% --- Visual Styling ---
\definecolor{courseblue}{RGB}{0, 51, 102}
\hypersetup{
    colorlinks=true,
    linkcolor=courseblue,
    urlcolor=blue,
}

% Code snippet styling
\definecolor{codegreen}{rgb}{0,0.6,0}
\definecolor{codegray}{rgb}{0.5,0.5,0.5}
\lstset{
    commentstyle=\color{codegreen},
    keywordstyle=\color{magenta},
    numberstyle=\tiny\color{codegray},
    stringstyle=\color{purple},
    basicstyle=\ttfamily\small,
    breaklines=true,
    frame=single,
    language=Python
}

% Custom Problem Box
\newtcolorbox{problem}[1]{
  colback=blue!5!white,
  colframe=courseblue,
  fonttitle=\bfseries,
  title=Problem #1
}

% --- Assignment Metadata ---
\newcommand{\courseName}{Mathematical Foundations of AI}
\newcommand{\courseURL}{https://eduenez.github.io/MathAIspring2026UTSA}
\newcommand{\assignmentTitle}{Homework Assignment \#0 (mock/template)}
\newcommand{\studentName}{Your Name Here} % <--- REPLACE THIS

% --- Header/Footer Setup ---
\pagestyle{fancy}
\fancyhf{}
\lhead{\courseName}
\rhead{\studentName}
\cfoot{\thepage}

% --- Document Starts Here ---
\begin{document}

\begin{center}
    {\LARGE \textbf{\assignmentTitle}} \\
    \vspace{5pt}
    {\large \href{\courseURL}{Course Site @GitHub}} \\
    \vspace{10pt}
    \textbf{Student:} \studentName \hfill \textbf{Date:} {30 February 0000}\\
    \rule{\linewidth}{0.4pt}
\end{center}

\section*{Part I: Mathematical Theory}

\begin{problem}{1: Linear Algebra \& Matrices}
  Define eigenvalues and eigenvectors of a matrix.
  Provide a $2 \times 2$ numerical example. 
\end{problem}

\begin{proof}[Solution]
An eigenvalue $\lambda$ (real or complex) and corresponding eigenvector $\mathbf{v}$ of a square matrix $A$ are any solutions to the equation:
\begin{equation}
    A\mathbf{v} = \lambda\mathbf{v},
  \end{equation}
  where the eigenvector $\mathbf{v} \ne \mathbf{0}$
  (the eigenvalue $\lambda$ may be zero).
Example:
\[
A = \begin{bmatrix}
4 & 1 \\
2 & 3
\end{bmatrix}
\]
has eigenvalue/eigenvector pairs $\lambda_1 = 2$, $\mathbf{v}_1 = (1, -2)$ and $\lambda_2 = 5$, $\mathbf{v}_2 = (1, 1)$.
\end{proof}

\begin{problem}{2: Probability and Lists}
List three common probability distributions used in machine learning and state Bayes' Theorem.
\end{problem}

\begin{proof}[Solution]
Common distributions include:
\begin{itemize}
    \item \textbf{Gaussian (Normal):} The “bell” curve.
    \item \textbf{Bernoulli:} For binary outcomes (yes/no, true/false, cat/dog).
    \item \textbf{Multinomial:} For categorical data (probability of observing each category out of a finite list).
\end{itemize}

Bayes' Theorem states that if $A, B$ are events such that $B$ has probability $\mathbb{P}(B)>0$, then the \emph{a priori} probability $\mathbb{P}(A)$ of the event $A$ and its \emph{a posteriori} conditional probability $\mathbb{P}(A|B)$ (of the conditonal event $A|B$, i.e., the probability that $A$ occurs given that $B$ has already occurred) are related by
\begin{equation} \label{eq:bayes}
    \mathbb{P}(A|B) = \frac{\mathbb{P}(A)\cdot\mathbb{P}(B|A)}{\mathbb{P}(B)},
  \end{equation}
  where $\mathbb{P}(B|A)$ is the reciprocal conditional probability of observing $B$ when $A$ is observed.
Equation \ref{eq:bayes} encapsulates the change (update) in beliefs after seeing new evidence.
\end{proof}

\section*{Part II: Code and Illustrations}
\begin{problem}{3: Python Implementation}
Write a simple Python function to calculate the dot product of two lists.
\end{problem}

\begin{proof}[Solution]
Below is the implementation of the dot product:
\begin{lstlisting}[language=Python]
def dot_product(vec_a, vec_b):
    """Returns the scalar product of two vectors."""
    return sum(a * b for a, b in zip(vec_a, vec_b))

# Example usage
print(dot_product([1, 2], [3, 4])) # Output: 11
\end{lstlisting}
\end{proof}

\begin{problem}{4: Figures and Visuals}
Include a placeholder for a figure and explain its importance.
\end{problem}

\begin{proof}[Solution]
Visualization is key for AI. For example, Figure \ref{fig:example} depicts a “loss curve”.

\begin{figure}[h]
    \centering
    \includegraphics[width=0.5\textwidth]{loss-curve.png}
    \caption{This is a placeholder for your model architecture or data plot.}
    \label{fig:example}
\end{figure}
\end{proof}

\end{document}
